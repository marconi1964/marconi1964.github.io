% Options for packages loaded elsewhere
\PassOptionsToPackage{unicode}{hyperref}
\PassOptionsToPackage{hyphens}{url}
%
\documentclass[
]{book}
\usepackage{amsmath,amssymb}
\usepackage{lmodern}
\usepackage{ifxetex,ifluatex}
\ifnum 0\ifxetex 1\fi\ifluatex 1\fi=0 % if pdftex
  \usepackage[T1]{fontenc}
  \usepackage[utf8]{inputenc}
  \usepackage{textcomp} % provide euro and other symbols
\else % if luatex or xetex
  \usepackage{unicode-math}
  \defaultfontfeatures{Scale=MatchLowercase}
  \defaultfontfeatures[\rmfamily]{Ligatures=TeX,Scale=1}
\fi
% Use upquote if available, for straight quotes in verbatim environments
\IfFileExists{upquote.sty}{\usepackage{upquote}}{}
\IfFileExists{microtype.sty}{% use microtype if available
  \usepackage[]{microtype}
  \UseMicrotypeSet[protrusion]{basicmath} % disable protrusion for tt fonts
}{}
\makeatletter
\@ifundefined{KOMAClassName}{% if non-KOMA class
  \IfFileExists{parskip.sty}{%
    \usepackage{parskip}
  }{% else
    \setlength{\parindent}{0pt}
    \setlength{\parskip}{6pt plus 2pt minus 1pt}}
}{% if KOMA class
  \KOMAoptions{parskip=half}}
\makeatother
\usepackage{xcolor}
\IfFileExists{xurl.sty}{\usepackage{xurl}}{} % add URL line breaks if available
\IfFileExists{bookmark.sty}{\usepackage{bookmark}}{\usepackage{hyperref}}
\hypersetup{
  pdftitle={關於 Rmarkdown / bookdown / blogdown / Hugodown},
  pdfauthor={Marconi Jiang},
  hidelinks,
  pdfcreator={LaTeX via pandoc}}
\urlstyle{same} % disable monospaced font for URLs
\usepackage{longtable,booktabs,array}
\usepackage{calc} % for calculating minipage widths
% Correct order of tables after \paragraph or \subparagraph
\usepackage{etoolbox}
\makeatletter
\patchcmd\longtable{\par}{\if@noskipsec\mbox{}\fi\par}{}{}
\makeatother
% Allow footnotes in longtable head/foot
\IfFileExists{footnotehyper.sty}{\usepackage{footnotehyper}}{\usepackage{footnote}}
\makesavenoteenv{longtable}
\usepackage{graphicx}
\makeatletter
\def\maxwidth{\ifdim\Gin@nat@width>\linewidth\linewidth\else\Gin@nat@width\fi}
\def\maxheight{\ifdim\Gin@nat@height>\textheight\textheight\else\Gin@nat@height\fi}
\makeatother
% Scale images if necessary, so that they will not overflow the page
% margins by default, and it is still possible to overwrite the defaults
% using explicit options in \includegraphics[width, height, ...]{}
\setkeys{Gin}{width=\maxwidth,height=\maxheight,keepaspectratio}
% Set default figure placement to htbp
\makeatletter
\def\fps@figure{htbp}
\makeatother
\setlength{\emergencystretch}{3em} % prevent overfull lines
\providecommand{\tightlist}{%
  \setlength{\itemsep}{0pt}\setlength{\parskip}{0pt}}
\setcounter{secnumdepth}{5}
\usepackage{booktabs}
\ifluatex
  \usepackage{selnolig}  % disable illegal ligatures
\fi
\usepackage[]{natbib}
\bibliographystyle{apalike}

\title{關於 Rmarkdown / bookdown / blogdown / Hugodown}
\author{Marconi Jiang}
\date{2021-03-21}

\begin{document}
\maketitle

{
\setcounter{tocdepth}{1}
\tableofcontents
}
\hypertarget{ux524dux8a00}{%
\chapter{前言}\label{ux524dux8a00}}

\textbf{Using R Markdown to build github.io static web pages}

無意間發現 merely-useful.github.io 網頁完美結合 github 及 github.io, 集合眾人之力, 內文集程式更新於 github, 並且以書籍形式發表於 github.io, 詢問作者 後, 得知是運用 R Markdown 的作品. 本以為是個簡單的 deployment, 實際執行後, 還是有不小的 efforts, 而且 google 到相關的文章不是太多, 只好自己努力.

\textbf{書中部分內容是作者的 try-and-error 實驗, 讀者可以直接略過這些經驗, 跳到第 §\ref{chap-summary} 章的成果}

相關連結

\begin{itemize}
\tightlist
\item
  Merely-useful 網站

  \begin{itemize}
  \tightlist
  \item
    \href{https://github.com/merely-useful/merely-useful.github.io}{github.com/merely-useful}\\
  \item
    \href{https://merely-useful.github.io}{merely-useful.github.io} 我喜歡 hyperlink 的設計 (本以為是 CSS 的能力, 發現是這網頁使用的是 \href{https://themes.gohugo.io/hugo-lithium-theme/}{Hugo 的 Lithium theme} )
  \end{itemize}
\item
  RMarkdown 的網站有兩個: {bookdown} 跟 {blogdown}, blogdown 似乎是比較舊, 但是含有安裝 Hugo 的說明, bookdown 比較新, 卻沒有 Hugo 的安裝說明

  \begin{itemize}
  \tightlist
  \item
    \href{https://bookdown.org/yihui/bookdown/get-started.html}{bookdown: Authoring Books and Technical Documents with R Markdown}\\
  \item
    \href{https://bookdown.org/yihui/blogdown/get-started.html}{blogdown: Creating Websites with R Markdown}
  \end{itemize}
\item
  勉強找到幾篇入門的教學文章, 試試看是否可行

  \begin{itemize}
  \tightlist
  \item
    這篇文章幫我完成了用 bookdown 編輯 .md 的文章格式, 然後可以快速 deploy 到 github.io 的理想, 請看 23 章 \href{https://www.math.pku.edu.cn/teachers/lidf/docs/Rbook/html/_Rbook/bookdown.html}{R 語言教學}\\
  \item
    這篇應該是完整且最新的詳細說明 \href{https://alison.rbind.io/post/new-year-new-blogdown/}{Getting any website up and running with all the moving parts (RStudio, GitHub, Hugo, Netlify)} (完成後發現, 她是以 blogdown 的格式, 也就是適合 blog 的文章, 跟我希望用的 bookdown 是完全不同的, 是否可以直接套用在 bookdown, 還需要再確定)\\
  \item
    \href{https://medium.com/@delucmat/how-to-publish-bookdown-projects-with-github-actions-on-github-pages-6e6aecc7331e}{How to publish bookdown projects with GitHub Actions on GitHub Pages}\\
  \item
    運用 Netlify 工具 deploy \href{https://www.garrickadenbuie.com/blog/blogdown-netlify-new-post-workflow/}{A Blogdown New Post Workflow with Github and Netlify}\\
  \item
    運用 Travis 工具 deploy 到 github \href{https://bookdown.org/yihui/blogdown/travis-github.html}{Blogdown - Travis + GitHub}
  \end{itemize}
\end{itemize}

\hypertarget{intro}{%
\chapter{安裝 R、RStudio 及 github 設定}\label{intro}}

\textbf{安裝 R、RStudio 套件}\\
就一般的套件安裝, 沒有特別需要注意的地方

\textbf{github 設定}\\
在 github.com 上新增一的 repo, repo 名稱必須是 marconijiang.github.io
不想破壞原有的 marconi1964.github.io 的設計, 重新申請 github 帳號, 並且新增一個 repo 叫做 marconijiang.github.io, 網頁位址為 \url{https://marconijiang.github.io}
- (這是 github 的標準設定, 只要設定 username.github.io 這個 repo, github 會自動設定 \url{https://username.github.io/} 這網頁導向這個 repo)

\hypertarget{ux958bux59cb-rmarkdown-ux5728-github.io-ux7684ux5192ux96aaux4e4bux65c5}{%
\chapter{開始 RMarkdown 在 github.io 的冒險之旅}\label{ux958bux59cb-rmarkdown-ux5728-github.io-ux7684ux5192ux96aaux4e4bux65c5}}

主要是根據 \href{https://alison.rbind.io/post/new-year-new-blogdown/}{Alison Hill 關於 blogdown 的 2021 更新} *1, (除非有其他的參考文章會特別標出)

\hypertarget{r-markdown-ux7684ux57faux672cux6307ux4ee4-2}{%
\section{R Markdown 的基本指令 *2}\label{r-markdown-ux7684ux57faux672cux6307ux4ee4-2}}

\begin{verbatim}
## 參考用, 待會兒執行時, 會有詳細指令說明
## 安裝套件 Install packages from CRAN

> if (!requireNamespace("remotes")) install.packages("remotes")  ## install remotes

## install blogdown from CRAN
> install.packages("blogdown") 
## Or, install blogdown from GitHub
> remotes::install_github("rstudio/blogdown")

## blogdown 是基於 Hugo (https://gohugo.io) 靜態網頁產生器, 需要  blogdown 下載並安裝 hugo
> blogdown::install_hugo()

## 安裝後, 如果要開啟 server 的 preview, 就執行以下指令
> blogdown::serve_site()
\end{verbatim}

\hypertarget{create-project-ref-1.-step-2-create-project}{%
\section{Create project (Ref 1. Step 2 Create project)}\label{create-project-ref-1.-step-2-create-project}}

\begin{itemize}
\tightlist
\item
  RStudio 選擇 File \textgreater{} New Project \textgreater{} Version Control \textgreater{} Git
\item
  複製 github 的 url, 我的例子是 \url{https://github.com/MarconiJiang/marconijiang.github.io.git}
\item
  Create Project
\end{itemize}

這時候, RStudio 會開個全新的畫面, 下載 marconijiang.github.io 的內容, 同時 file 目錄指向筆電 local repo 的目錄

\hypertarget{create-site-ref-1.-step-3-create-site}{%
\section{Create site (Ref 1. Step 3 Create site)}\label{create-site-ref-1.-step-3-create-site}}

\hypertarget{ux5b89ux88dd-blogdown}{%
\subsection{安裝 blogdown}\label{ux5b89ux88dd-blogdown}}

\begin{verbatim}
## 到 RStudio 的 console 下, 執行以下指令
> if (!requireNamespace("remotes")) install.packages("remotes")
> remotes::install_github("rstudio/blogdown")

##  完成後, 會出現安裝完成字樣
##  * DONE (blogdown)
\end{verbatim}

\hypertarget{ux5728-rstudio-blogdown-ux4e0bux5275ux5efaux65b0ux7684ux7db2ux7ad9}{%
\subsection{在 RStudio / blogdown 下創建新的網站}\label{ux5728-rstudio-blogdown-ux4e0bux5275ux5efaux65b0ux7684ux7db2ux7ad9}}

執行以下指令

\begin{verbatim}
> library(blogdown)
> blogdown::new_site(theme = 'yihui/hugo-lithium')
\end{verbatim}

原始文章安裝的是另一個主題 (theme), 我嘗試的做了些修改, 都是出現錯誤後, 發現 error message 裡面嘗試去 github 尋找 lithium, 因此我 google 了 ``github lithium theme'', 找到了至少 3 個 github 包含 lithium 這個主題 (theme), 但是, 只有其中一個的 README.md 說明清楚如何在 blogdown 上安裝, 就照著他的說明進行

\begin{itemize}
\tightlist
\item
  \href{https://github.com/yihui/hugo-lithium}{Yihui Hugo Lithium} \#\# 清楚說明, 雖然是從 \href{https://github.com/jrutheiser/hugo-lithium-theme}{jrutheiser} fork 出來
\item
  \href{https://github.com/jrutheiser/hugo-lithium-theme}{Jrutheiser Hugo Lithium} \#\# 應該是原始出處
\item
  \href{https://github.com/janikvonrotz/hugo-lithium-theme}{janikvonrotz Hugo Lithium}
\item
  \href{https://themes.gohugo.io/hugo-lithium-theme/}{Hugo Lithium 官網}
\end{itemize}

\begin{verbatim}
## 以下為錯誤嘗試, 不必重複
> new_site(theme = "wowchemy/starter-academic")   ## 原始文章作法

也參考了 https://bookdown.org/yihui/blogdown/other-themes.html  的說明
> blogdown::new_site(theme = "lxndrblz/anatole")

> new_site(theme = "Lithium")       ##  我的錯誤嘗試
> new_site(theme = "Hugo/Lithium")  ##  我的錯誤嘗試
\end{verbatim}

需要再按 `y' 來確定執行後, RStudio 就開始創建一個新的網站, 有三個方式可以看到網頁內容(雖然目前內容是空的)
1. RStudio 右下方的 Viewer 視窗(在 Files 的右邊)
2. Viewer 視窗下, 有個箭頭的 icon, 意思是 Show in new window, 按下後, 會開 browser 的新視窗
3. 自行到 browser 下, 輸入網址 \url{http://localhost:4321} (實際位址可能會有變動, 需查看 RStudio 視窗內容顯示的網址, 類似如下的內容)

\begin{verbatim}
― Creating your new site
| Installing the theme yihui/hugo-lithium from github.com
trying URL 'https://github.com/yihui/hugo-lithium/archive/master.tar.gz'
Content type 'application/x-gzip' length unknown
downloaded 109 KB

| Adding the sample post to content/post/2020-12-01-r-rmarkdown/index.Rmd
| Converting all metadata to the YAML format
| Adding netlify.toml in case you want to deploy the site to Netlify
| Adding .Rprofile to set options() for blogdown
― The new site is ready
○ To start a local preview: use blogdown::serve_site(), or the RStudio add-in "Serve Site"
○ To stop a local preview: use blogdown::stop_server(), or restart your R session
> Want to serve and preview the site now? (y/n) y
Launching the server via the command:
  C:/Users/mjiang/AppData/Roaming/Hugo/0.81.0/hugo.exe server --bind 127.0.0.1 -p 4321 --themesDir themes -t hugo-lithium -D -F --navigateToChanged
Serving the directory . at http://localhost:4321
Launched the hugo server in the background (process ID: 32508). To stop it, call blogdown::stop_server() or restart the R session.
Rendering content/post/2020-12-01-r-rmarkdown/index.Rmd... Done.
\end{verbatim}

網頁到此已順利創建完成, 接下來就是網頁內容的修改, 及如何 deploy 到 github.io

\hypertarget{ux5c07-rmarkdown-blogdown-ux5167ux5bb9-deploy-ux5230-netlify}{%
\chapter{將 RMarkdown blogdown 內容 deploy 到 Netlify}\label{ux5c07-rmarkdown-blogdown-ux5167ux5bb9-deploy-ux5230-netlify}}

參考 \href{https://alison.rbind.io/post/new-year-new-blogdown/\#using-github}{Ref. 1 Using github 章節}

\hypertarget{ux8a2dux5b9a-.gitignore-ux96d6ux7136ux4e0dux77e5ux9053ux9019ux6642ux5019ux8a2dux5b9a-.gitignore-ux7684ux5fc5ux8981ux6027-ux9084ux662fux7167ux8457ux8aaaux660eux505a}{%
\section{設定 .gitignore : 雖然不知道這時候設定 .gitignore 的必要性, 還是照著說明做}\label{ux8a2dux5b9a-.gitignore-ux96d6ux7136ux4e0dux77e5ux9053ux9019ux6642ux5019ux8a2dux5b9a-.gitignore-ux7684ux5fc5ux8981ux6027-ux9084ux662fux7167ux8457ux8aaaux660eux505a}}

\begin{verbatim}
file.edit(".gitignore")
\end{verbatim}

RStudio 會將 cursor 移到左上角的檔案內容檢視視窗, 在該視窗類加上以下內容

\begin{verbatim}
## 原始內容 
.Rproj.user
.Rhistory
.RData
.Ruserdata
## 增加以下內容
.DS_Store
Thumbs.db
\end{verbatim}

\hypertarget{blogdown-ux6aa2ux67e5-check_gitignore}{%
\section{blogdown 檢查 check\_gitignore()}\label{blogdown-ux6aa2ux67e5-check_gitignore}}

執行第一次 commit 之前, 先用 blogdown 檢查

\begin{verbatim}
> blogdown::check_gitignore()
\end{verbatim}

不太清楚 {[}TODO{]} 再這裡的意義, 待研究

\begin{verbatim}
― Checking .gitignore
| Checking for items to remove...
○ Nothing to see here - found no items to remove.
| Checking for items to change...
○ Nothing to see here - found no items to change.
| Checking for items you can safely ignore...
○ Found! You have safely ignored: .DS_Store, Thumbs.db
| Checking for items to ignore if you build the site on Netlify...
● [TODO] When Netlify builds your site, you can safely add to .gitignore: /public/, /resources/
| Checking for files required by blogdown but not committed...
● [TODO] Found 1 file that should be committed in GIT:

  layouts/shortcodes/blogdown/postref.html
― Check complete: .gitignore
\end{verbatim}

\hypertarget{blogdown-ux6aa2ux67e5-check_content}{%
\section{blogdown 檢查 check\_content()}\label{blogdown-ux6aa2ux67e5-check_content}}

還不清楚這動作的意義, 待確認

\begin{verbatim}
> blogdown::check_content()
\end{verbatim}

執行結果

\begin{verbatim}
― Checking content files
| Checking for validity of YAML metadata in posts...
○ All YAML metadata appears to be syntactically valid.
| Checking for previewed content that will not be published...
○ Found 0 files with future publish dates.
○ Found 0 files marked as drafts.
| Checking your R Markdown content...
○ All R Markdown files have been knitted.
○ All R Markdown output files are up to date with their source files.
| Checking for .html/.md files to clean up...
○ Found 0 duplicate .html output files.
○ Found 0 incompatible .html files to clean up.
| Checking for the unnecessary 'content/' directory in theme...
○ Great! Your theme does not contain the content/ directory.
― Check complete: Content
\end{verbatim}

\hypertarget{ux5728-netlify-ux7db2ux7ad9ux4e0aux5efaux7f6eux4e26ux4e14-publish-site}{%
\section{在 Netlify 網站上建置並且 Publish site}\label{ux5728-netlify-ux7db2ux7ad9ux4e0aux5efaux7f6eux4e26ux4e14-publish-site}}

參考 \href{https://alison.rbind.io/post/new-year-new-blogdown/\#step-5-publish-site}{Ref 1. Publish site 章節}

我們在這章節會使用比較高端的做法, 讓 Netlify 來將我們建立的網站, 建置到 github 上.
1. 到 \href{https://www.netlify.com/}{Netlify} 網站
2. 用 github 的帳號去 sign up (無須建立新帳號)
3. log in 後, 選擇 New site from Git \textgreater{} Continuous Deployment: GitHub.
4. (optional) 到 (Netlify) menu 右上角 Site Setting 點選後, 更改 Change site name 到自己已歡的名字, 我挑選的是 marconijiang, 所以我的網站就是 \url{https://marconijiang.netlify.app}. 如果修改, 會是 Netlify 自行設定的網址名稱, 位於左上角類似 xxxx-yyyyy-zzzz.netlify.app 的網址
5. 可以另開一個 browser 頁面, 輸入網址 marconijiang.netlify.app, 就可以看到自己的 github 上的內容 (這時候的內容是 github 網站的內容, 如果還沒跟電腦上的同步, 內容可能是空的)

\hypertarget{git-ux6307ux4ee4ux5c07ux96fbux8166ux7684ux5167ux5bb9ux66f4ux65b0ux65bc-github}{%
\section{git 指令將電腦的內容更新於 github}\label{git-ux6307ux4ee4ux5c07ux96fbux8166ux7684ux5167ux5bb9ux66f4ux65b0ux65bc-github}}

移到 RStudio 左下角的 terminal (console 右邊), 透過一般的 git 指令去更新 github 上的內容

\begin{verbatim}
## 這時候, 目錄位址應該是從網頁內容 git clone 下來的目錄, 都是 xxxx.github.io
## 像我的目錄就是 marconijiang.github.io 
$ git add *
$ git commit -m "init commit"
$ git push origin main      # 最近, github 將 "master" branch 改成 main
##  需要輸入 github 的 user id 跟 password
\end{verbatim}

這時候, 再回到 browser 的 \url{https://marconijiang.netlify.app} 稍稍等個一分鐘, 網頁會自動更新到最新的內容

\hypertarget{ux91cdux65b0ux5b89ux88dd-bookdown}{%
\chapter{重新安裝 bookdown}\label{ux91cdux65b0ux5b89ux88dd-bookdown}}

此章節主要參考 Ref 7. \href{https://github.com/r-lib/hugodown}{rib-hugodown github 網頁}

\hypertarget{ux5b89ux88dd-bookdown-ux5957ux4ef6}{%
\section{安裝 bookdown 套件}\label{ux5b89ux88dd-bookdown-ux5957ux4ef6}}

\begin{verbatim}
> install.packages("devtools")
## devtools::install_github('rstudio/bookdown')
> devtools::install_github("r-lib/hugodown")

## 不知是否是版本問題, 還是改安裝了 0.66.0 版後才沒問題
> hugodown::hugo_install('0.66.0')
\end{verbatim}

\hypertarget{ux5728-rstudio-ux5efaux7acbux65b0ux5c08ux6848}{%
\section{在 RStudio 建立新專案}\label{ux5728-rstudio-ux5efaux7acbux65b0ux5c08ux6848}}

進入 RStudio -\textgreater{} New Project -\textgreater{} New Directory -\textgreater{} New Project -\textgreater{} ``Directory name'' -\textgreater{} ``Create Project''

\hypertarget{ux5275ux5efa-hugo-ux7684-academic-ux6a21ux677fux7db2ux7ad9}{%
\section{創建 hugo 的 academic 模板網站}\label{ux5275ux5efa-hugo-ux7684-academic-ux6a21ux677fux7db2ux7ad9}}

創建 hugo 的 academic 模板網站

\begin{verbatim}
> hugodown::create_site_academic()
\end{verbatim}

開始 hugodown 網頁的 server

\begin{verbatim}
> hugodown::hugo_start()
Starting server on port 1313
\end{verbatim}

\hypertarget{chap-summary}{%
\chapter{Summary - 成功安裝 bookdown, 且順利 deploy 到 github.io}\label{chap-summary}}

\hypertarget{ux5b89ux88ddux5957ux4ef6}{%
\section{安裝套件}\label{ux5b89ux88ddux5957ux4ef6}}

我只是安裝了 blogdown 套件, 就自動安裝了 bookdown 套件(應該是如此)

\begin{verbatim}
> if (!requireNamespace("remotes")) install.packages("remotes")  
> remotes::install_github("rstudio/blogdown")
\end{verbatim}

\hypertarget{ux7522ux751f-bookdown-ux7bc4ux4f8bux6a94ux6848}{%
\section{產生 bookdown 範例檔案}\label{ux7522ux751f-bookdown-ux7bc4ux4f8bux6a94ux6848}}

到 RStudio -\textgreater{} File -\textgreater{} New project -\textgreater{} New directory -\textgreater{} 向下找到 ``Book project using bookdown'' -\textgreater{} Directory 的名字為 marconijiang.github.io 即可產生帶有範例的 bookdown 程式在 marconijiang.github.io 目錄下.

\hypertarget{deploy-ux5230-github.io}{%
\section{Deploy 到 github.io}\label{deploy-ux5230-github.io}}

到 RStudio 右上方, Environment 同一頁面的 Build, 選擇 Build Book 下的 bookdown::gitbook, RStudio 就開始將內容轉至 html, 儲存於 \_book 子目錄下, 這時候到 \url{https://marconijiang.github.io/_book/index.html} 會出現 404 not found

\hypertarget{ux7db2ux9801ux51faux73fe-404-not-foundux6642-ux95dcux65bc-.nojekyll-ux7684ux89e3ux6c7aux65b9ux5f0f}{%
\section{網頁出現 404 not found時, 關於 .nojekyll 的解決方式}\label{ux7db2ux9801ux51faux73fe-404-not-foundux6642-ux95dcux65bc-.nojekyll-ux7684ux89e3ux6c7aux65b9ux5f0f}}

github pages 是基於 \href{https://jekyllrb.com/docs/github-pages/}{jekyll} 支援的網頁, 因此, 如果以底線 (underline) 開頭的檔案或目錄, github 不會將這目錄上傳, 詳見\href{https://github.blog/2009-12-29-bypassing-jekyll-on-github-pages/}{Bypassing Jekyll on GitHub Pages}, 偏偏 bookdown 標準設定的 html 輸出目錄正是 \_book, 以底線開頭的目錄, 因此我們有兩個作法:\\
1. 在 檔案 \_bookdown.yml 修改輸出的目錄名稱, 詳見第 §\ref{chap-bookdown-outputdir} \_bookdown.yml 檔案說明;\\
2. 新增 .nojekyll 檔案, 作法如下:

\begin{itemize}
\tightlist
\item
  在 index.html 的同一個目錄下新增 .nkjekyll 這檔案, 這檔案不必有內容, 一般都是用 linux 的指令 touch 來創建該檔案
\end{itemize}

\begin{verbatim}
$ touch .nojekyll
\end{verbatim}

新增沒有內容的檔案, 名字為 .nojekyll 的檔案到 \_book 目錄下, 詳細內容參考 \href{https://stackoverflow.com/questions/11577147/how-to-fix-http-404-on-github-pages}{github 404 not found due to missing .nojekyll} 就可以成功看到 \url{https://marconijiang.github.io/_book/index.html} 的內容

\hypertarget{chap-bookdown}{%
\chapter{快速使用 Bookdown 的說明}\label{chap-bookdown}}

需要回頭看第 §\ref{chap-summary} 章的 bookdown 安裝, 再繼續往下看

\hypertarget{ux958bux59cbux5168ux65b0ux7684-bookdown-ux6a94ux6848}{%
\section{開始全新的 bookdown 檔案}\label{ux958bux59cbux5168ux65b0ux7684-bookdown-ux6a94ux6848}}

\begin{enumerate}
\def\labelenumi{\arabic{enumi}.}
\tightlist
\item
  我會 copy 這篇文章/目錄 (github.com/marconi1964/About\_bookdown) 的全部內容到新建的目錄
\item
  打開 RStudio
\item
  到 RStudio 右下角的 file, 找到新建目錄
\item
  找到並點擊 marconi1964.github.Rproj, 此時 RStudio 會重新建立以這個目錄內容的新 project, 就可以開始了
\end{enumerate}

\hypertarget{ux6ce8ux610fux4e8bux9805}{%
\section{注意事項}\label{ux6ce8ux610fux4e8bux9805}}

\begin{itemize}
\tightlist
\item
  如果是 post 在 gibhub.io, 需要在 index.html 的同一個目錄下新增 .nkjekyll 這檔案, 這檔案不必有內容, 一般都是用 linux 的指令 touch 來創建該檔案
\end{itemize}

\begin{verbatim}
$ touch .nojekyll
\end{verbatim}

\hypertarget{chap-bookdown-outputdir}{%
\section{檔案結構}\label{chap-bookdown-outputdir}}

我在 github 設定的 檔案目錄結構如下

\begin{verbatim}
github.com/marconi1964
|
|-- About_bookdown         # 書的 Rmarkdown 內容儲存於 github 的第一層子目錄下
    |-- _bookdown.yml      # 需要新增一行到此檔案, 將 html 輸出到 marconi1964.github.io 下一層的目錄, 這目錄名稱可以自訂, 此例子為 about_bookdown, 後續 bookdown 輸出後, 可以在 https://marconi1964.github.io/about_bookdown/index.html 看到此書的內容 
         output_dir: "../marconi1964.github.io/about_bookdown"
    |-- index.Rmd          # bookdown 會依照檔案名稱安排章節順序
    |-- 01-intro.Rmd       # 順序依序為 : index.Rmd, 01-intro.Rmd, 02-xxx.Rmd
    |-- 02-xxx.Rmd         # 數字後面的檔案名稱只是讓作者/讀者了解內容
 
|-- JetsonNano_book        # 書的 Rmarkdown 內容儲存於 github 的第一層子目錄下
    |-- _bookdown.yml      # 需要新增一行到此檔案, 將 html 輸出到 marconi1964.github.io 下一層的目錄, 這目錄名稱可以自訂, 此例子為 JetsonNano, 後續 bookdown 輸出後, 可以在 https://marconi1964.github.io/JetsonNano/index.html 看到此書的內容
         output_dir: "../marconi1964.github.io/JetsonNano"
    |-- ...
    
|-- marconi1964.github.io  # html 的內容儲存於此目錄下
    |-- index.html         # 需自行手工創建, 建立 hyperlink 連接到不同的書內容
    |-- about_bookdown     # 此目錄及內容由 bookdown 創建及產生, 可以在 https://marconi1964.github.io/about_bookdown/index.html 看到此書的內容
        |-- .nojekyll      # 唯有這個檔案需要自行創建
        |-- index.html
        |-- ...
    |-- JetsonNano         # 此目錄及內容由 bookdown 創建及產生, 可以在 https://marconi1964.github.io/JetsonNano/index.html 看到此書的內容
        |-- .nojekyll      # 唯有這個檔案需要自行創建
        |-- index.html   
        |-- ...
    |--
\end{verbatim}

\hypertarget{rmd-ux5167ux5bb9ux6ce8ux610fux4e8bux9805}{%
\section{Rmd 內容注意事項}\label{rmd-ux5167ux5bb9ux6ce8ux610fux4e8bux9805}}

我目前的想法只是單純的使用 md 檔案格式, 而沒運用到 Rmd 的功能, 這樣, 就可以直接轉換到 python 常用的 md 格式

其它待補充

\hypertarget{ux53c3ux8003ux6587ux4ef6ux7684ux53c3ux7167}{%
\section{參考文件的參照}\label{ux53c3ux8003ux6587ux4ef6ux7684ux53c3ux7167}}

待補充

\hypertarget{ux5716ux8868ux7684ux53c3ux7167}{%
\section{圖表的參照}\label{ux5716ux8868ux7684ux53c3ux7167}}

待補充

參考資料\\
1. \url{https://alison.rbind.io/post/new-year-new-blogdown/}\\
2. \url{https://bookdown.org/yihui/blogdown/installation.html}

Lithium 相關主題\\
3. \href{https://github.com/yihui/hugo-lithium}{Yihui Hugo Lithium}\\
4. \href{https://github.com/jrutheiser/hugo-lithium-theme}{Jrutheiser Hugo Lithium} \#\# 應該是 Lithium 原始出處\\
5. \href{https://github.com/janikvonrotz/hugo-lithium-theme}{janikvonrotz Hugo Lithium}\\
6. \href{https://themes.gohugo.io/hugo-lithium-theme/}{Hugo Lithium 官網}

Hugodown 相關網頁\\
7. \href{https://github.com/r-lib/hugodown}{r-lib/hugodown github 官網}

Bookdown 相關網頁\\
8. \href{https://www.math.pku.edu.cn/teachers/lidf/docs/Rbook/html/_Rbook/bookdown.html}{R 語言教學} - 這篇文章幫我完成了用 bookdown 編輯 .md (實際上是 .rmd 但是盡量只用到 .md 的功能) 的文章格式, 然後可以快速 deploy 到 github.io 的理想, 請看 23 章

  \bibliography{book.bib,packages.bib}

\end{document}
